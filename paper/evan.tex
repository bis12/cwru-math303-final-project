\section{Background}

\subsection{Who was Mersenne?}
Marin Mersenne was born on September 8, 1588 near Oize, Sarthe, France. He maintained
strong interests in the sciences, philosophy, music, and religion throughout the course of his
life. At the age of 23, he joined the Roman Catholic Church and later became a friar within
the French Order of the Minims. During a period in which scientific ideas were repressed by
the Church, Mersenne played a crucial role in the diffusion of mathematical and scientific
information throughout Western Europe. He met frequently in Paris with his academic circle
to share the latest theories and discoveries, and to maintain discourse with the likes of Galileo,
Christiaan Huygens, Pierre de Fermat, and more. In 1644, Mersenne published his most famous
work, Cogita Physico-Mathematica, in which he proposed a formula for generating prime
numbers. These prime numbers became known as Mersenne Primes.

\subsection{What is a Mersenne Prime?}
A Mersenne number is a power of two less one, of the form
\begin{align}
M = 2^p - 1
\end{align}
The integer p must itself be a prime number in order to generate a Mersenne prime. If p is a
composite number, a composite number will be generated. This can be shown by...
%TODO: ADD PROOF that composite powers will create composite mersennes

\subsection{The Mersenne Conjecture}

The first of three Mersenne Conjectures appeared in Mersenne’s 1644 publication of Cogita
Physico-Mathematica, in which Mersenne postulated that primes are generated for p = 2, 3, 5,
7, 13, 17, 19, 31, 67, 127, and 257. As it turns out, Mersenne missed three primes for which his
formula generates a prime, namely 61, 89, and 107. He also missed the fact that the primes 67
and 257 do not generate primes.

The second Mersenne Conjecture was proposed 1989 by P. T. Bateman, J. L. Selfridge and
Wagstaff, Jr., S. S. in the article “The New Mersenne Conjecture.” This conjecture states that if
two of the three following conditions are true, then the third must also hold true

\begin{enumerate}
\item $p = 2^k \pm 1$ or $p = 4k \pm 3$ (where $k$ is a natural number)
\item $2^p – 1$ is a prime (more specifically, a Mersenne Prime)
\item $(2^p – 1)/3$ is a prime
\end{enumerate}

As it stands, the New Mersenne Prime Conjecture has yet to be disproven.

The third and most recent Mersenne Conjecture is the Lenstra–Pomerance–Wagstaff
Conjecture. They propose that there are infinitely many primes, and that the number of
Mersenne Primes less than a number x may be approximated by the following formula

%TODO: get his formulas
\begin{align}
\end{align}

Where the exponential of gamma, called the Euler-Mascheroni constant, is approximately
1.78107.
