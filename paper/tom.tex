%%%%%%%%%%%%%%%%%%%%%%%%%%%%%%%%%%%%%%%%%%%%%%%%%%%%%%%%%%%%%%%
\section{Properties Of Mersenne Numbers}
%%%%%%%%%%%%%%%%%%%%%%%%%%%%%%%%%%%%%%%%%%%%%%%%%%%%%%%%%%%%%%%

Mersenne Primes have a few properties that make them especially interesting. Primarily, they
are trophy hunted for their size, but they also have some relation to other number theory topics.

%%%%%%%%%%%%%%%%%%%%%%%%%%%%%%%%%%%%%%%%%%%%%%%%%%%%%%%%%%%%%%%
\subsection{If $2^p-1$ is prime, then $p$ is prime}
%%%%%%%%%%%%%%%%%%%%%%%%%%%%%%%%%%%%%%%%%%%%%%%%%%%%%%%%%%%%%%%

If $2^p-1$ is a prime, then so too will be $p$~\cite{LighNeal}. And, contrapositively, if $p$ is composite,
so too will be $2^p-1$. If $p$ is composite, it can be written $p = ab$ ($a,b \ne 1$). It is easy to notice, then,
that $2^{ab} - 1 = (2^a)^b - 1$ factors into 
\begin{equation*}
    2^{ab} - 1 = (2^a-1)(2^{a(b-1)} + 2^{a(b-2)} \ldots 2^{a} + 1)
\end{equation*}

This is a useful property in reducing the search space for Mersenne Primes. Although Mersenne Prime hunters
(i.e. the GIMPS project) only search for prime $p$, it is still no easy task to determine the primality
of the exponents on the order of $10^{7}$.

%%%%%%%%%%%%%%%%%%%%%%%%%%%%%%%%%%%%%%%%%%%%%%%%%%%%%%%%%%%%%%%
\subsection{Mersenne Primes and Even Perfect Numbers}
%%%%%%%%%%%%%%%%%%%%%%%%%%%%%%%%%%%%%%%%%%%%%%%%%%%%%%%%%%%%%%%

Mersenne numbers have a one-to-one correspondence to even perfect numbers~\cite{perfect}. A perfect number has
a sum of its proper divisors equal to the number itself. For instance, the first three perfect numbers
are 6 ($1 + 2 + 3$), 28 ($1 + 2 + 4 + 7 + 14$), and 496 ($1 + 2 + 4 + 8 + 16 + 31 + 62 + 124 + 248$).

Perfect numbers can be generated by Mersenne Numbers. Given a Mersenne Prime $M_p$, a perfect number can be
generated with the equation
\begin{equation}
\label{eqn:perfno}
2^{p-1}\cdot(2^p-1)
\end{equation}

The first three perfect numbers are generated with $M_2$, $M_3$, $M_5$, and the fourth is generated with $M_7$ by computing 
\[ 2^6\cdot(2^7-1) = 64\cdot127 = 8128 \]

Furthermore, Euler proved all perfect numbers must be generated with this equation~\cite{perfect}.
