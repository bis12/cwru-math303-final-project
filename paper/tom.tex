%%%%%%%%%%%%%%%%%%%%%%%%%%%%%%%%%%%%%%%%%%%%%%%%%%%%%%%%%%%%%%%
\section{Mersenne Numbers}
%%%%%%%%%%%%%%%%%%%%%%%%%%%%%%%%%%%%%%%%%%%%%%%%%%%%%%%%%%%%%%%

Mersenne Primes have a few properties that make them especially interesting. Primarily, they
are trophy hunted for their size, but they also have some relation to other number theory topics.

%%%%%%%%%%%%%%%%%%%%%%%%%%%%%%%%%%%%%%%%%%%%%%%%%%%%%%%%%%%%%%%
\subsection{The Mersenne Conjecture}
%%%%%%%%%%%%%%%%%%%%%%%%%%%%%%%%%%%%%%%%%%%%%%%%%%%%%%%%%%%%%%%
The first of three Mersenne Conjectures appeared in Mersenne’s 1644 publication of 
\textit{Cogita Physico-Mathematica}, in which Mersenne postulated that primes are generated for p = 2, 3, 5,
7, 13, 17, 19, 31, 67, 127, and 257. As Mersenne's conjecture was made with limited computational
resources, it turns out this his list is not entirely correct.
\begin{figure}[h]
\begin{tabular}{rcccccccccccccc}
\textbf{Mersenne}&2&3&5&7&13&17&19&31&&67&&&127&257\\
\textbf{Corrected}&2&3&5&7&13&17&19&31&61&&89&107&127&
\end{tabular}
\caption{The primes that Mersenne found and missed.}
\label{fig:conj}
\end{figure}

Mersenne missed three primes ($p = 61, 89, and~107$) and included two composites ($p = 67, 257$). Mersenne's list and the corrected list can be found in Figure \ref{fig:conj}.

How did Mersenne arrive at his list, then? A possible explanation, coincident with Mersenne's description in
\textit{Cogita}, is that
\begin{equation*}
    M_p \text{ is prime if and only if $p$ is a prime of one of the forms } 2^k\pm1 \text{ or } 2^{2k}\pm3
\end{equation*}
except that Mersenne did not include $p = 2^{6}-3 = 61$ in his list! Recent authors have
suggested this was an unintentional omission ~\cite{newconjecture}.


%%%%%%%%%%%%%%%%%%%%%%%%%%%%%%%%%%%%%%%%%%%%%%%%%%%%%%%%%%%%%%%
\subsection{The New Mersenne Conjecture}
%%%%%%%%%%%%%%%%%%%%%%%%%%%%%%%%%%%%%%%%%%%%%%%%%%%%%%%%%%%%%%%

Regardless, in 1989, P. T. Bateman, J. L. Selfridge and Wagstaff, Jr., S. published a revised Mersenne Conjecture.
The \textit{New Mersenne Conjecture} seeks to clearly define the properties of Mersenne and Wagstaff primes\footnote{A Wagstaff prime is a number of the form $(2^p-1)/3$}.
In their paper describing The New Mersenne Conjecture, Bateman, Selfridge, and Wagstaff conjecture that if two of: 
\begin{enumerate}
\item $p = 2^k \pm 1$ or $p = 4k \pm 3$ (where $k$ is a natural number)
\item $2^p - 1$ is a prime (more specifically, a Mersenne Prime)
\item $(2^p - 1)/3$ is a prime
\end{enumerate}
are true, then the third must also hold true.

Currently, the New Mersenne Prime Conjecture has yet to be proven or disproven although extensive testing
has shown the New Mersenne Conjecture to be true at least through $p < 100,000$.

%%%%%%%%%%%%%%%%%%%%%%%%%%%%%%%%%%%%%%%%%%%%%%%%%%%%%%%%%%%%%%%
\subsection{The Lenstra-Pomerance-Wagstaff Conjecture}
%%%%%%%%%%%%%%%%%%%%%%%%%%%%%%%%%%%%%%%%%%%%%%%%%%%%%%%%%%%%%%%

The third (and most recent) Mersenne Conjecture is the \textit{Lenstra-Pomerance-Wagstaff Conjecture}. 
Lenstra, Pomerance, and Wagstaff propose that there are infinitely many Mersenne primes, and furthermore
that the number of Mersenne Primes less than a number x may be approximated by the following formula~\cite{utm.edu-heuristic}

\begin{align}
e^{\gamma} \times \log \left( \dfrac{\log{x}}{\log{2}} \right)
\end{align}

Where $\gamma$, called the Euler-Mascheroni constant, is approximately 0.57722.
%%%%%%%%%%%%%%%%%%%%%%%%%%%%%%%%%%%%%%%%%%%%%%%%%%%%%%%%%%%%%%%
\subsection{Mersenne Primes and Even Perfect Numbers}
%%%%%%%%%%%%%%%%%%%%%%%%%%%%%%%%%%%%%%%%%%%%%%%%%%%%%%%%%%%%%%%

Mersenne numbers have a one-to-one correspondence to even perfect numbers~\cite{perfect}. A perfect number has
a sum of its proper divisors equal to the number itself. For instance, the first three perfect numbers
are 6 ($1 + 2 + 3$), 28 ($1 + 2 + 4 + 7 + 14$), and 496 ($1 + 2 + 4 + 8 + 16 + 31 + 62 + 124 + 248$).

Perfect numbers can be generated by Mersenne Numbers. Given a Mersenne Prime $M_p$, a perfect number can be
generated with the equation
\begin{equation}
\label{eqn:perfno}
2^{p-1}\cdot(2^p-1)
\end{equation}

The first three perfect numbers are generated with $M_2$, $M_3$, $M_5$, and the fourth is generated with $M_7$ by computing 
\[ 2^6\cdot(2^7-1) = 64\cdot127 = 8128 \]

Furthermore, Euler proved all perfect numbers must be generated with this equation~\cite{perfect}.

%%%%%%%%%%%%%%%%%%%%%%%%%%%%%%%%%%%%%%%%%%%%%%%%%%%%%%%%%%%%%%%
\subsection{Mersenne Numbers are Repunits}
%%%%%%%%%%%%%%%%%%%%%%%%%%%%%%%%%%%%%%%%%%%%%%%%%%%%%%%%%%%%%%%
A repunit is a number made up entirely of the digit 1 in any base system.  In base 2 (binary) all Mersenne numbers will be repunits because they are one less than a power of 2.  In binary, a power of 2 will be a 1 followed by all 0's.  So subtracting one will flip all of those 0's into 1's resulting in a repunit.
\begin{align*}
2^{10} = 1024 = (10000000000)_2\\
2^{10} - 1 = 1023 = (1111111111)_2
\end{align*}
