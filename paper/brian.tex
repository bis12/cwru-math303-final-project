%%%%%%%%%%%%%%%%%%%%%%%%%%%%%%%%%%%%%%%%%%%%%%%%%%%%%%%%%%%%%%%
\section{Finding Mersenne Primes}
%%%%%%%%%%%%%%%%%%%%%%%%%%%%%%%%%%%%%%%%%%%%%%%%%%%%%%%%%%%%%%%

%TODO: talk about the history of finding Mersenne Primes, including the badass...
%TODO: ...fact that two Mersenne Primes were found the first day of using computers. ~\cite{Robinson54}

The advent of the modern computer has allowed us to find larger and larger Mersenne Primes.  The process is fairly straightforward from a high level, but the implementation details are of great importance and interest. The Great Internet Mersenne Prime Search (GIMPS) has been the most successful project in this regard and we will examine their process as an example.  The description of their process comes directly from GIMPS \cite{gimps} in a section on how the math behind the search works. 

First we will note that GIMPS is a distributed computing project.  There is a master server that organizes all of the clients that volunteers run on their own personal computers.  Each of these clients accomplishes a task that is appropriate for their processing power.  %TODO: finish this line of thought and then go into process from THE MATH page  of gimps site

%%%%%%%%%%%%%%%%%%%%%%%%%%%%%%%%%%%%%%%%%%%%%%%%%%%%%%%%%%%%%%%
\subsection{Lucas-Lehmer Test}
%%%%%%%%%%%%%%%%%%%%%%%%%%%%%%%%%%%%%%%%%%%%%%%%%%%%%%%%%%%%%%%

Mersenne Primes are interesting in their own right, but the reason that we use them in projects like GIMPS is the Lucas-Lehmer test.  This test makes testing the primality of massive primes possible, and somehow remains simple to understand how it works.  The harder part is understanding why it works, but we'll present a proof that is fairly convincing from Rosen's \emph{A Proof of the Lucas-Lehmer Test}\cite{rosen}.


\begin{thm} 
Let $p$ be a prime other than 2 and define a sequence $L_n$ by the rule
\begin{align}
L_0 = 4,&&L_{n+1} = L_n^2 - 2 \pmod{2^p - 1}
\end{align}
Then $2^p - 1$ is prime iff $L_{p-2} = 0$.
\end{thm}

As mentioned before, the sequence is simple to understand 

\begin{proof}
Define two new sequences
\begin{align}
U_0 = 0,& U_1 = 1,& U_{n+1} = 4U_n - U_{n-1}\\ 
V_0 = 2,& V_1 = 4,& V_{n+1} = 4V_n - V_{n-1}
\end{align}
\end{proof}



%%%%%%%%%%%%%%%%%%%%%%%%%%%%%%%%%%%%%%%%%%%%%%%%%%%%%%%%%%%%%%%
\subsubsection{Algorithm}
%%%%%%%%%%%%%%%%%%%%%%%%%%%%%%%%%%%%%%%%%%%%%%%%%%%%%%%%%%%%%%%

The sequence definition above is sufficient if we merely wish to understand what the algorithm does, but does not explicitly show each step involved.  We can redefine it in pseudocode very similar to what would actually be programmed into a computer that is testing for primality and then use it to try a couple of examples. We can also try to analyze the algorithm to see how efficient it is.

\begin{codebox}
\Procname{$\proc{LLT}(prime)$}
\li$s = 4$
\li$M = 2^p - 1$
\li \For $i \gets 0$ \To $p - 2$
\Do\li
$s = (s^2 - 2) \pmod{M}$
\End\li
\If $s == 0$
\Then
\li\Return True \li
\Else
\li\Return False \li
\End
\end{codebox}

%%%%%%%%%%%%%%%%%%%%%%%%%%%%%%%%%%%%%%%%%%%%%%%%%%%%%%%%%%%%%%%
\subsubsection{Example with prime}
%%%%%%%%%%%%%%%%%%%%%%%%%%%%%%%%%%%%%%%%%%%%%%%%%%%%%%%%%%%%%%%

We can use the Mersenne Prime 7 to show how the algorithm executes when it is run on a prime.
\begin{tabular}{ll}
step 0&s = 4, $M = 2^7 -1 = 63$
\end{tabular}

%%%%%%%%%%%%%%%%%%%%%%%%%%%%%%%%%%%%%%%%%%%%%%%%%%%%%%%%%%%%%%%
\subsubsection{Example with composite} 
%%%%%%%%%%%%%%%%%%%%%%%%%%%%%%%%%%%%%%%%%%%%%%%%%%%%%%%%%%%%%%%

We can use 11 to show how the algorithm executes when it is run on a composite number.


%%%%%%%%%%%%%%%%%%%%%%%%%%%%%%%%%%%%%%%%%%%%%%%%%%%%%%%%%%%%%%%
\section{Closing Notes}
%%%%%%%%%%%%%%%%%%%%%%%%%%%%%%%%%%%%%%%%%%%%%%%%%%%%%%%%%%%%%%%

Mersenne Primes are not only studied to satisfy mathematical curiosity.  They have been found to be useful in a number of different applications.  

%%%%%%%%%%%%%%%%%%%%%%%%%%%%%%%%%%%%%%%%%%%%%%%%%%%%%%%%%%%%%%%
\subsection{Mersenne Twister}
%%%%%%%%%%%%%%%%%%%%%%%%%%%%%%%%%%%%%%%%%%%%%%%%%%%%%%%%%%%%%%%

%%%%%%%%%%%%%%%%%%%%%%%%%%%%%%%%%%%%%%%%%%%%%%%%%%%%%%%%%%%%%%%
\subsection{Maybe We Can Do Better}
\label{sec:knuth}
%%%%%%%%%%%%%%%%%%%%%%%%%%%%%%%%%%%%%%%%%%%%%%%%%%%%%%%%%%%%%%%
In 1980 Knuth published the second volume of \emph{The Art of Computer Programming}, which contains the following claim.

\begin{quote}
The world’s largest explicitly known prime numbers have always been Mersenne primes ... But the situation will probably change soon, since Mersenne primes are getting harder to find, and since [the exercises present] an efficient test for primes of other forms. \cite{taocp}
\end{quote}

%CONTINUE HERE WITH THE STUFF FROM KNUTH
